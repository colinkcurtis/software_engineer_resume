%-----------------------------------------------------------------------------------------------------------------------------------------------%
%	The MIT License (MIT)
%
%	Copyright (c) 2021 Jitin Nair
%
%	Permission is hereby granted, free of charge, to any person obtaining a copy
%	of this software and associated documentation files (the "Software"), to deal
%	in the Software without restriction, including without limitation the rights
%	to use, copy, modify, merge, publish, distribute, sublicense, and/or sell
%	copies of the Software, and to permit persons to whom the Software is
%	furnished to do so, subject to the following conditions:
%	
%	THE SOFTWARE IS PROVIDED "AS IS", WITHOUT WARRANTY OF ANY KIND, EXPRESS OR
%	IMPLIED, INCLUDING BUT NOT LIMITED TO THE WARRANTIES OF MERCHANTABILITY,
%	FITNESS FOR A PARTICULAR PURPOSE AND NONINFRINGEMENT. IN NO EVENT SHALL THE
%	AUTHORS OR COPYRIGHT HOLDERS BE LIABLE FOR ANY CLAIM, DAMAGES OR OTHER
%	LIABILITY, WHETHER IN AN ACTION OF CONTRACT, TORT OR OTHERWISE, ARISING FROM,
%	OUT OF OR IN CONNECTION WITH THE SOFTWARE OR THE USE OR OTHER DEALINGS IN
%	THE SOFTWARE.
%	
%
%-----------------------------------------------------------------------------------------------------------------------------------------------%

%----------------------------------------------------------------------------------------
%	DOCUMENT DEFINITION
%----------------------------------------------------------------------------------------

% article class because we want to fully customize the page and not use a cv template
\documentclass[a4paper,12pt]{article}

%----------------------------------------------------------------------------------------
%	FONT
%----------------------------------------------------------------------------------------

% % fontspec allows you to use TTF/OTF fonts directly
% \usepackage{fontspec}
% \defaultfontfeatures{Ligatures=TeX}

% % modified for ShareLaTeX use
% \setmainfont[
% SmallCapsFont = Fontin-SmallCaps.otf,
% BoldFont = Fontin-Bold.otf,
% ItalicFont = Fontin-Italic.otf
% ]
% {Fontin.otf}

%----------------------------------------------------------------------------------------
%	PACKAGES
%----------------------------------------------------------------------------------------
\usepackage{url}
\usepackage{parskip} 	

%other packages for formatting
\RequirePackage{color}
\RequirePackage{graphicx}
\usepackage[usenames,dvipsnames]{xcolor}
\usepackage[scale=0.9]{geometry}

%tabularx environment
\usepackage{tabularx}

%for lists within experience section
\usepackage{enumitem}

% centered version of 'X' col. type
\newcolumntype{C}{>{\centering\arraybackslash}X} 

%to prevent spillover of tabular into next pages
\usepackage{supertabular}
\usepackage{tabularx}
\newlength{\fullcollw}
\setlength{\fullcollw}{0.47\textwidth}

%custom \section
\usepackage{titlesec}				
\usepackage{multicol}
\usepackage{multirow}

%CV Sections inspired by: 
%http://stefano.italians.nl/archives/26
\titleformat{\section}{\Large\scshape\raggedright}{}{0em}{}[\titlerule]
\titlespacing{\section}{0pt}{10pt}{10pt}

%for publications
\usepackage[style=authoryear,sorting=ynt, maxbibnames=2]{biblatex}

%Setup hyperref package, and colours for links
\usepackage[unicode, draft=false]{hyperref}
\definecolor{linkcolour}{rgb}{0,0.2,0.6}
\hypersetup{colorlinks,breaklinks,urlcolor=linkcolour,linkcolor=linkcolour}
\addbibresource{citations.bib}
\setlength\bibitemsep{1em}

%for social icons
\usepackage{fontawesome5}

%debug page outer frames
%\usepackage{showframe}


%----------------------------------------------------------------------------------------
%	BEGIN DOCUMENT
%----------------------------------------------------------------------------------------
\begin{document}

% non-numbered pages
\pagestyle{empty}

%----------------------------------------------------------------------------------------
%	TITLE
%----------------------------------------------------------------------------------------

% \begin{tabularx}{\linewidth}{ @{}X X@{} }
% \huge{Your Name}\vspace{2pt} & \hfill \emoji{incoming-envelope} email@email.com \\
% \raisebox{-0.05\height}\faGithub\ username \ | \
% \raisebox{-0.00\height}\faLinkedin\ username \ | \ \raisebox{-0.05\height}\faGlobe \ mysite.com  & \hfill \emoji{calling} number
% \end{tabularx}

\begin{tabularx}{\linewidth}{@{} C @{}}
\Huge{Colin Curtis} \\[7.5pt]
\href{https://github.com/colinkcurtis}{\raisebox{-0.05\height}\faGithub\ colinkcurtis} \ $|$ \
\href{https://gitlab.com/colinkcurtis}{\raisebox{-0.05\height}\faGitlab\ colinkcurtis} \ $|$ \
\href{https://www.linkedin.com/in/colinkcurtis/}{\raisebox{-0.05\height}\faLinkedin\ Colin Curtis} \ $|$ \
\href{mailto:colinkcurtis@gmail.com}{\raisebox{-0.05\height}\faEnvelope \ colinkcurtis@gmail.com} \ $|$ \
\href{919-525-7837}{\raisebox{-0.05\height}\faMobile \ 919-525-7837} \\
\end{tabularx}

%----------------------------------------------------------------------------------------
% SUMMARY
%----------------------------------------------------------------------------------------

%Interests/ Keywords/ Summary
\section{Professional Summary}
Software engineering is a complex and fast-moving profession where solid communication and team-work is as important as technical skill. My focus is in back-end, data-intensive systems with a dev-ops mindset. I am constantly looking for improved processes and effective automations.


%----------------------------------------------------------------------------------------
%	EDUCATION
%----------------------------------------------------------------------------------------
\section{Education}
\begin{tabularx}{\linewidth}{@{}l X@{}}
2012 - 2014 M.S. (Physics) at \textbf{North Carolina State University} \hfill \normalsize \\
2007 - 2012 B.S. (Physics w/ Math Minor) at \textbf{Appalachian State University} \hfill

\end{tabularx}

%----------------------------------------------------------------------------------------
%	OTHER SKILLS & INTERESTS
%----------------------------------------------------------------------------------------
\section{Skills}
\begin{tabularx}{\linewidth}{@{}l X@{}}
Other Skills: &  \normalsize{Linear Algebra, Calculus, Statistical Mechanics, Error/Uncertainty Analysis, Fractal Geo.}\\
Interests:  &  \normalsize{Gardening, Cooking, Mountain Biking, Skiing, Running, Sci-fi, History}\\
\end{tabularx}


%----------------------------------------------------------------------------------------
% WORK EXPERIENCE
%----------------------------------------------------------------------------------------


%Experience
\section{Work Experience}

\begin{tabularx}{\linewidth}{ @{}l r@{} }
\textbf{Senior Software Engineer - Garner Health} & \hfill August 2022 - February 2023 \\[3.75pt]
\multicolumn{2}{@{}X@{}}{
\begin{minipage}[t]{\linewidth}
    \begin{itemize}[nosep,after=\strut, leftmargin=1em, itemsep=3pt]

        \item Systems and Tech Stack:
        		\begin{itemize}
        			\item Event-driven, micro-services architecture, RESTful (OpenAPI spec), cloud-native
        			\item Typical Layers: API, Daemon, Service, Model, Test
			\item Python 3.10: flask, boto3, sqlalchemy, logging, pytest, alembic, connexion, asyncio, behave, gnupg
			\item AWS: S3, EC2, ECR, EKR, Step Functions, Lambda, Cognito, Transfer Family, Secrets Manager
			\item Infrastructure-as-code: Docker, Kubernetes, Terraform
			\item pair-programming with GPT-3 in the openAI playground
   		\end{itemize}

	\item Responsibilities:
		\begin{itemize}
		    	\item Writing code for new features and bug fixes
			\item Requirements gathering, feature planning, and project scheduling
    			\item Code review
			\item System design
    			\item Dev-ops for the cloud-native system
    			\item Support/on-call rotations
		\end{itemize}

    	\item Projects and Work Contributed:
		\begin{itemize}
    			\item Successfully designed and implemented a PGP decryption module for the file ingestion system (this system was composed of a set of AWS Step Functions kicked off by 				an AWS Lambda driven by a microservice daemon when a client file was received in one of our S3 buckets)
			\item Designed and implemented versioning of configuration files for transformation of ingested data
			\item SQL (sqlalchemy) query optimizations
			\item System testing of APIs and expected workflows using Behavior Driven Design (Given, When, Then) and the Python 'behave' library
			\item Application of Linear Algebra principles to database schema design (minimization of necessary dimensions and relationships)
   			\item Extensively reviewed code for other engineers with a high acceptance rate for my recommended revisions
    			\item Contributed to and revised Product Requirement Documents for engineering/management communications
		\end{itemize}


    \end{itemize}
    \end{minipage}
}
\end{tabularx}

\begin{tabularx}{\linewidth}{ @{}l r@{} }
\textbf{Software Engineer - Actalent Services} & \hfill April 2019 - August 2022 \\[3.75pt]
\multicolumn{2}{@{}X@{}}{
\begin{minipage}[t]{\linewidth}
    \begin{itemize}[nosep,after=\strut, leftmargin=1em, itemsep=3pt]

    \item Systems and Tech Stack:
    	\begin{itemize}
    		\item Micro-services for on-premises-hosted, bespoke tire-design CAD and analysis web application
    		\item Python 3, react.js, numpy, logging, pytest
    	\end{itemize}

    \item Projects and Work Contributed
    	\begin{itemize}
    		\item Designed and implemented a Dynamometer Dashboard for engine test data analysis at Ford Motors
    		\item Contributed to Bridgestone tire design CAD/analysis system written in Python 3 and React.js
		\begin {itemize}
			\item Microservices architecture hosted using on-premises resources
			\item Added 'overlays' to the CAD system, allowing tire engineers to view a transparency of one tire design overlaid on to another
			\item Refactored the automatic Excel report generation - tire engineers click a button in the CAD system and the Python back-end generates a downloadable report
    			\item Converted and upgraded FMAT, a tire image analysis tool, from MATLAB to Python 3 and introduced 'alpha-shapes' computational geometry technique to improve the 				boundary estimates of complex 2-D shapes
		\end{itemize}
    	\end{itemize}
    \end{itemize}
    \end{minipage}
}
\end{tabularx}

\begin{tabularx}{\linewidth}{ @{}l r@{} }
\textbf{Research Software Engineer - RENCI @ UNC - CH} & \hfill June 2018 - March 2019 \\[3.75pt]
\multicolumn{2}{@{}X@{}}{
\begin{minipage}[t]{\linewidth}
    \begin{itemize}[nosep,after=\strut, leftmargin=1em, itemsep=3pt]
    	\item NIH National Center for Advancing Translational Sciences (NCATS) - Biomedical Data Translator Project - Green Team
    	\item Wrote, maintained, and deployed APIs to give access to bioinformatics data hosted in-house using Marathon, Mezos, Github, Jenkins, Docker, and nginx
    	\item Jupyter notebooks used extensively to analyze and share data
    \end{itemize}
    \end{minipage}
}
\end{tabularx}

\begin{tabularx}{\linewidth}{ @{}l r@{} }
\textbf{Research Assistant, Krim Group} & \hfill January 2015 - April 2018 \\[3.75pt]
\multicolumn{2}{@{}X@{}}{
\begin{minipage}[t]{\linewidth}
    \begin{itemize}[nosep,after=\strut, leftmargin=1em, itemsep=3pt]
    	\item Fully designed and developed ALAI, a MATLAB application for automating fractal analysis of nanoscopic images
	\begin{itemize}
		\item Reduced user's active analysis time, per image, by a factor of \textasciitilde 50
		\item Classification of 'fitting zones' (linear v. exponential) by $Adjusted-R^2$ comparison between multiple fitting attempts
	\end{itemize}
    	\item Primary skills: equipment building, data capture, data analysis, and mathematical modeling
    	\item Subject-matter expertise: carbon nano-structures, inter-facial friction at the atomic scale
    \end{itemize}
    \end{minipage}
}
\end{tabularx}

\begin{tabularx}{\linewidth}{ @{}l r@{} }
\textbf{Research Assistant, Clarke Group} & \hfill January 2015 - April 2018 \\[3.75pt]
\multicolumn{2}{@{}X@{}}{
\begin{minipage}[t]{\linewidth}
    \begin{itemize}[nosep,after=\strut, leftmargin=1em, itemsep=3pt]
    	\item Research focus: Polymers, LASER for spectroscopy and photothermal heating, nanoparticle synthesis
    	\item Utilized ANSYS Maxwell mesh-calculation to simulate 3-D electro-magnetic fields
    	\item Used and maintained LabVIEW software systems for instrument control and data collection
    \end{itemize}
    \end{minipage}
}
\end{tabularx}


%----------------------------------------------------------------------------------------
%	PUBLICATIONS
%----------------------------------------------------------------------------------------
\section{Publications}
\begin{tabularx}{\linewidth}{ @{}l r@{} }
\multicolumn{2}{@{}X@{}}{
\begin{minipage}[t]{\linewidth}
    	\begin{itemize}[nosep,after=\strut, leftmargin=1em, itemsep=3pt]

    		\item First Author, \textit{A Comparative Study of the Nanoscale and Macroscale Attributes... of Nanodiamonds}, Beilstein Journal of Nanotechnology, Sep 2017 (PDF available 			here: https://www.beilstein-journals.org/bjnano/content/pdf/2190-4286-8-205.pdf)
    		\item Second Author, \textit{Unconfined, melt edge electrospinning from multiple, spontaneous, self-organized polymer jets}, Materials Research Express, 28 Nov 2014 (Vol. 1, 			Num. 4)
		\item Third Author, \textit{A Tribological Study of $\gamma$-Fe$_{2}O_{3}$ Nanoparticles in Aqueous Suspension}, Tribology Letters, Dec 2018 (66:130)
	\end{itemize}
\end{minipage}

}\end{tabularx}



\vfill
\center{\footnotesize Last updated: \today}

\end{document}